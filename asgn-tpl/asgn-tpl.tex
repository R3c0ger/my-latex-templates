% compile with: pdflatex *2
\documentclass{article}
\usepackage{asgn}

\title{\textbf{Title of the Assignment Template \\ Assignment 1}}
\author{Your Name | Student No.: 7355608 | Email: example@email.com}
\date{\today}

\newcommand{\q}[1]{\qst{#1}[Problem ][][Roman][2]} % Custom question command

\begin{document}
\maketitle
%%%%%%%%%%%%%%%%%%%%%%%%%%%%%%%%%%%%%%%

\section{Common Commands}

\couriertext{\textbackslash couriertext} \texttt{\textbackslash texttt} $\norm{\argmin\limits_{x}\vect{M}\inv}$

\subsection{Questions}

\qst{
	\begin{sub}
		\item This is sub item one in \texttt{sub} environment.
		\item This is sub item two.
		\item This is sub item three.
	\end{sub}

	\begin{quote}
		This is a \texttt{quote} environment inside the question box.
	\end{quote}
}

\qstnt{
	This is a question box without title.
	
	\[
		\begin{dcases}
			\alpha = \dfrac{\sum\limits_{i=1}^{n}(x_i - \bar{x})(y_i - \bar{y})}{\sum\limits_{i=1}^{n}(x_i - \bar{x})^2} \\
			\beta = \bar{y} - \alpha\bar{x}
		\end{dcases}
	\]
}

\setcounter{qstcounter}{5}
% \qst{
\q{
	\begin{itemize}
		\item \texttt{\textbackslash qst\{blabla\}[Problem ][][Roman][2]}
		\item blabla
	\end{itemize}
% }[Problem ][][Roman][2]
}

\subsection{Answers}

\sln This is the solution using \texttt{\textbackslash sln}.

\sln* This is the solution using \texttt{\textbackslash sln*}. There is a new line after the prefix.

\ans This is the answer using \texttt{\textbackslash ans}.

\ans* This is the answer using \texttt{\textbackslash ans*}. There is a new line after the prefix.

\prf{
	This is the proof environment content using \texttt{\textbackslash prf\{\}}.
	\par There is a square at the end of the proof.
}

\section{Figures, Tables, and Code Listings}

\subsection{Figures}

Use custom \texttt{\textbackslash img} command to insert image (shown as Figure~\ref{fig:img_example}):

\img[0.5][H]{img/exp-img.jpg}
[Example Image. This image displays the ReLU (Rectified Linear Unit) activation function, which is defined as \( \text{ReLU}(x) = \max(0, x) \). It is widely used in neural networks due to its simplicity and effectiveness in introducing non-linearity.]
[fig:img_example][2cm]

\begin{lstlisting}[language=tex]
\img
[0.5]             % Scale (as a fraction of text width)
[H]               % Placement
{img/exp-img.jpg} % Image file
[Example Image]   % Caption
[fig:img_example] % Label
[2cm]             % Margin
\end{lstlisting}

Use custom \texttt{\textbackslash tikzimg} command to insert TikZ image (in a question box below):

\qstnt{
\tikzimg[
	neuron/.style={circle, draw, minimum size=1cm},
	input/.style={circle, draw, minimum size=0.8cm},
	>=stealth
]{
	% Input nodes
	\node[input] (a) at (0,2) {$a$};
	\node[input] (b) at (0,0) {$b$};
	\node[input] (c) at (0,-2) {$c$};
	% Perceptron node
	\node[neuron] (perceptron) at (3,0) {$\text{step}(\cdot)$};
	% Bias
	\node[above=0.7cm] at (perceptron) {$\theta = 1$};

	% Connections with weights
	\draw[-latex] (a) -- node[above=3pt] {$-1$} (perceptron);
	\draw[-latex] (b) -- node[above=3pt] {$+2$} (perceptron);
	\draw[-latex] (c) -- node[above=3pt] {$-1$} (perceptron);
	
	% Output
	\draw[-latex] (perceptron) -- (5,0) node[right] {Output};
}
}

\subsection{Tables}

Use \texttt{\textbackslash tbl} command to insert tables (shown as Table~\ref{tab:truth_table}):

\tbl{
    \begin{tabular}{|c c c|c c|c|}
        \hline
        $a$ & $b$ & $c$ & $\neg a$ & $\neg c$ & $\neg a \lor b \lor \neg c$ \\
        \hline
        0 & 0 & 0 & 1 & 1 & 1 \\
        0 & 0 & 1 & 1 & 0 & 1 \\
        0 & 1 & 0 & 1 & 1 & 1 \\
        0 & 1 & 1 & 1 & 0 & 1 \\
        1 & 0 & 0 & 0 & 1 & 1 \\
        1 & 0 & 1 & 0 & 0 & 0 \\
        1 & 1 & 0 & 0 & 1 & 1 \\
        1 & 1 & 1 & 0 & 0 & 1 \\
        \hline
    \end{tabular}
}[Truth table for the boolean expression $\neg a \lor b \lor \neg c$][tab:truth_table]

\begin{lstlisting}[language=tex]
\tbl
[H]                       % Placement
{                         % Table content
	\begin{tabular}{cols} % You can change the env you need
		...
	\end{tabular}
}
[Example Table]           % Caption
[tab:example]             % Label
\end{lstlisting}

\subsection{Code Listings}

Inline code:

\lstinline[language=python]{print("Hello, World!")}

Block code (imported from external files):

\lstinputlisting[language=python]{code/exp-code.py}

\lstinputlisting[language=bash, basicstyle=\small\ttfamily]{code/exp-bash.sh}

Block code:

\begin{lstlisting}[
	language=tex, 
	caption={Example LaTeX Code}, 
	label={lst:example_latex},
	% style=framestyle  % Custom style as follows
	basicstyle=\small\ttfamily, 
	columns=flexible, 
	numbers=none, 
	backgroundcolor=\color{white}, 
	frame=single
]
\begin{lstlisting}[
	language=tex,
	caption={Example LaTeX Code},
	label={lst:example_latex},
	% style=framestyle  % Custom style as follows
	basicstyle=\small\ttfamily, 
	columns=flexible, 
	numbers=none, 
	backgroundcolor=\color{white}, 
	frame=single
]
	...
\end{...}
\end{lstlisting}

\subsection{Code Listings Using Custom Commands}

Inline code using \texttt{\textbackslash code} command: \code[python]{print("Hello, World!")}

Block code that imports from external files using \texttt{\textbackslash codef} command (shown as Listing~\ref{lst:exp_code}):

\codef{code/exp-code.py}[Example Python Code][lst:exp_code]

\end{document}