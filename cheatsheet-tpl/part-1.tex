\section{Example Section}

\subsection{Example Subsection}

\subsubsection{Example Subsubsection}

This is a template for cheatsheets that compiles with \textbf{PdfLaTeX}. 

Please refer to the \texttt{cheatsheet.sty} file for style settings and the \texttt{cheatsheet.tex} or PDF file for the main document structure. In this cheatsheet, we demonstrate various features provided by the template.

\section{Basic Text Formatting}

This is \textbf{bold} text, this is \textit{italic} text (\emph{emph} text), this is \underline{underlined} text, this is \textsc{small caps} text, and this is \texttt{monospaced} text.

We use the \texttt{enumitem} package to make \texttt{itemize} lists and \texttt{enumerate} lists more compact.

\begin{itemize}
  \item Item 1
  \item Item 2
  \begin{itemize}
    \item Item 2.1
    \item Item 2.2
    \begin{itemize}
      \item Item 2.2.1
      \item Item 2.2.2
    \end{itemize}
  \end{itemize}
  \item Item 3
\end{itemize}

\begin{enumerate}
  \item First
  \item Second
  \begin{enumerate}
    \item Second.A
    \item Second.B
    \begin{enumerate}
      \item Second.B.i
      \item Second.B.ii
    \end{enumerate}
  \end{enumerate}
  \item Third
\end{enumerate}

\section{Math Mode Example}

Inline math: $\text{Attention}(Q, K, V) = \text{Softmax}(\frac{QK^{\top}}{\sqrt{d_k}})V$.

Display math:
\[
  \text{Attention}(Q, K, V) = \text{Softmax}\left(\frac{QK^{\top}}{\sqrt{d_k}}\right)V
\]

\texttt{align} environment:
\begin{align*}
  \text{MultiHead}(Q, K, V) &= \textrm{Concat}(\text{head}_1, \dots, \text{head}_h)W^O\\
  \text{where head}_i &= \text{Attention}(QW_i^Q, KW_i^K, VW_i^V)
\end{align*}

\texttt{equation} environment with numbering:
\begin{equation}
  \text{FFN}(x) = \max(0, xW_1 + b_1)W_2 + b_2
\end{equation}

\texttt{dcases} environment:
\[
  f(x) =
  \begin{dcases}
    e^{-1/x}, & x > 0\\
    0, & x \leq 0
  \end{dcases}
\]

Equation in box using \texttt{csbox} command:
\csbox{equation}{
  \text{Attention}(Q, K, V) = \text{Softmax}\left(\frac{QK^{\top}}{\sqrt{d_k}}\right)V
}
\begin{lstlisting}[language=tex]
\csbox{equation}{
  \text{Attention}(Q, K, V) = \text{Softmax}\left(\frac{QK^{\top}}{\sqrt{d_k}}\right)V
}
\end{lstlisting}

\section{Code Listings}

\begin{itemize}
  \item Inline code (without colors): \\
  \texttt{\textbackslash texttt\{print("Hello World")\}}
  \item Inline code (with colors, default Python): \\
  \texttt{\textbackslash code\{print("Hello World")\}} \\
  \texttt{\textbackslash code[python]\{print("Hello World")\}} \\
  \code{print("Hello World")}
  \item Code block using \texttt{lstlisting} environment:
  \begin{lstlisting}[language=c]
#include <stdio.h>
int main() {
    printf("Hello, World!\n");
    return 0;
}
  \end{lstlisting}
  \item Code block using \texttt{codef} command to include from files (default Python):
  \codef[python]{code-example.py}
  \begin{lstlisting}[language=tex]
\codef{code-example.py}
% \codef[python]{code-example.py}
  \end{lstlisting}
\end{itemize}

\section{Tables \& Figures}

Use \texttt{cstable} to make tables.

\cstable{c X}{
  \toprule
  \textbf{Header 1} & \textbf{Header 2} \\
  \midrule
  Row 1, Col 1 & This is some text in column 2 that will automatically wrap to fit the column width. \\
  \midrule
  Row 2, Col 1 & Another row with more text to demonstrate text wrapping in the table cell. \\
  \bottomrule
}

\begin{lstlisting}[language=tex]
\cstable{c X}{
  \toprule
  \textbf{Header 1} & \textbf{Header 2} \\
  \midrule
  Row 1, Col 1 & This is some text in column 2 that will automatically wrap to fit the column width. \\
  \midrule
  Row 2, Col 1 & Another row with more text to demonstrate text wrapping in the table cell. \\
  \bottomrule
}
\end{lstlisting}

Use \texttt{csfig} to include figures.

\csfig{fig-example.png}

\begin{lstlisting}[language=tex]
\csfig{fig-example.png}
\end{lstlisting}

\section{Others}

This is an example of a TODO note: \\\TODOS{Finish writing this section.}

TikZ example:

\centering

\begin{tikzpicture}[
    node distance=0.6cm and 1.2cm,
    neuron/.style={circle, draw, minimum size=0.6cm, thick, inner sep=1pt},
    arrow/.style={-Stealth, semithick},
    label/.style={align=center, font=\small}
]

% Input layer
\node[neuron] (x1) {$x_1$};
\node[neuron, below=0.2cm of x1] (x2) {$x_2$};
\node[below=0.1cm of x2, label] {Input\\Layer};

% Hidden layer 1
\node[neuron, right=of x1] (h12) {$h_2^1$};
\node[neuron, above=0.2cm of h12] (h11) {$h_1^1$};
\node[neuron, below=0.2cm of h12] (h13) {$h_3^1$};
\node[neuron, below=0.2cm of h13] (h14) {$h_4^1$};
\node[below=0.1cm of h14, label] {Hidden\\Layer 1};

% Hidden layer 2
\node[neuron, right=of h11] (h21) {$h_1^2$};
\node[neuron, below=0.2cm of h21] (h22) {$h_2^2$};
\node[neuron, below=0.2cm of h22] (h23) {$h_3^2$};
\node[neuron, below=0.2cm of h23] (h24) {$h_4^2$};
\node[below=0.1cm of h24, label] {Hidden\\Layer 2};

% Output layer
\path (h22) -- (h23) coordinate[midway] (middle);
\node[neuron, right=of middle] (y) {$\hat{y}$};
\node[below=0.1cm of y, label] {Output\\Layer};

% Bias nodes
\node[neuron, above=0.2cm of h11, fill=gray!20] (b1) {$b_1$};
\node[neuron, above=0.2cm of h21, fill=gray!20] (b2) {$b_2$};
\node[neuron, above=0.2cm of y, fill=gray!20] (b3) {$b_3$};

% Connections
% Input layer to first hidden layer connections (partial weight annotations)
\foreach \i in {1,2} {
    \foreach \j in {1,2,3,4} {
        \draw[arrow] (x\i) -- (h1\j);
    }
}
\draw[arrow] (x1) -- node[above, pos=0.3, font=\tiny] {$w_{11}^1$} (h11);
\draw[arrow] (x2) -- node[below, pos=0.3, font=\tiny] {$w_{24}^1$} (h14);

% Hidden layer 1 to hidden layer 2 connections (partial weight annotations)
\foreach \i in {1,2,3,4} {
    \foreach \j in {1,2,3,4} {
        \draw[arrow] (h1\i) -- (h2\j);
    }
}
\draw[arrow] (h11) -- node[above, pos=0.4, font=\tiny] {$w_{11}^2$} (h21);
\draw[arrow] (h14) -- node[below, pos=0.4, font=\tiny] {$w_{44}^2$} (h24);

% Hidden layer 2 to output layer connections
\foreach \i in {1,2,3,4} {
    \draw[arrow] (h2\i) -- (y);
}
\foreach \i in {1,2,3,4} {
    \draw[arrow] (h2\i) -- node[above, pos=0.3, font=\tiny] {$w_{\i}^3$} (y);
}

% Bias connections
\foreach \i in {1} {
    \draw[arrow, dashed] (b1) -- (h1\i);
    \draw[arrow, dashed] (b2) -- (h2\i);
}
\draw[arrow, dashed] (b3) -- (y);

% Bias labels
\node[above=1pt of b1, font=\tiny] {Bias};
\node[above=1pt of b2, font=\tiny] {Bias};
\node[above=1pt of b3, font=\tiny] {Bias};

\end{tikzpicture}

3D Plot using TikZ:

\tdplotsetmaincoords{60}{110}

\begin{tikzpicture}[scale=1, tdplot_main_coords,
    axis/.style={->, thick},
    point/.style={circle, fill=red, inner sep=1.5pt}]
    
    \draw[axis] (0,0,0) -- (3.5,0,0) node[anchor=north east] {$x$};
    \draw[axis] (0,0,0) -- (0,3.5,0) node[anchor=north west] {$y$};
    \draw[axis] (0,0,0) -- (0,0,2.5) node[anchor=south] {$z$};
    
    \def\amplitude{2.2}
    \def\xcenter{1.5}
    \def\ycenter{1.5}
    \def\xspread{0.7}
    \def\yspread{0.7}
    
    \foreach \i in {0,0.02,...,1} {
        \foreach \j in {0,0.02,...,1} {
            \pgfmathsetmacro{\x}{3*\i}
            \pgfmathsetmacro{\y}{3*\j}
            \pgfmathsetmacro{\z}{\amplitude*exp(-((\x-\xcenter)^2/(2*\xspread^2) + (\y-\ycenter)^2/(2*\yspread^2)))}
            
            \pgfmathsetmacro{\intensity}{100*\z/\amplitude}
            \fill[blue!\intensity!green, opacity=0.8] (\x,\y,\z) circle (0.4pt);
        }
    }
    
    \fill[red] (\xcenter,\ycenter,\amplitude) circle (2pt) 
        node[above, font=\small] {Peak: $(\mu_x,\mu_y)$};

\end{tikzpicture}

\raggedright